% ****** Start of file apssamp.tex ******
%
%   This file is part of the APS files in the REVTeX 4.1 distribution.
%   Version 4.1r of REVTeX, August 2010
%
%   Copyright (c) 2009, 2010 The American Physical Society.
%
%   See the REVTeX 4 README file for restrictions and more information.
%
% TeX'ing this file requires that you have AMS-LaTeX 2.0 installed
% as well as the rest of the prerequisites for REVTeX 4.1
%
% See the REVTeX 4 README file
% It also requires running BibTeX. The commands are as follows:
%
%  1)  latex apssamp.tex
%  2)  bibtex apssamp
%  3)  latex apssamp.tex
%  4)  latex apssamp.tex
%
\documentclass[%
 reprint,
%superscriptaddress,
%groupedaddress,
%unsortedaddress,
%runinaddress,
%frontmatterverbose, 
%preprint,
%showpacs,preprintnumbers,
%nofootinbib,
%nobibnotes,
%bibnotes,
 amsmath,amssymb,
 aps,
%pra,
%prb,
%rmp,
%prstab,
%prstper,
%floatfix,
]{revtex4-1}

\usepackage{graphicx}% Include figure files
\usepackage{dcolumn}% Align table columns on decimal point
\usepackage{bm}% bold math
%\usepackage{hyperref}% add hypertext capabilities
%\usepackage[mathlines]{lineno}% Enable numbering of text and display math
%\linenumbers\relax % Commence numbering lines

%\usepackage[showframe,%Uncomment any one of the following lines to test 
%%scale=0.7, marginratio={1:1, 2:3}, ignoreall,% default settings
%%text={7in,10in},centering,
%%margin=1.5in,
%%total={6.5in,8.75in}, top=1.2in, left=0.9in, includefoot,
%%height=10in,a5paper,hmargin={3cm,0.8in},
%]{geometry}

\begin{document}

%\preprint{APS/123-QED}

\title{How to Write an Outline}% Force line breaks with \\
\thanks{Notes of \em{Whitesides' Group: Writing a Paper} }%

\author{Determine the authors}


%\date{\today}% It is always \today, today,
             %  but any date may be explicitly specified

\begin{abstract}
Do not write abstract until the end.

\end{abstract}

%\pacs{Valid PACS appear here}% PACS, the Physics and Astronomy
                             % Classification Scheme.
%\keywords{Suggested keywords}%Use showkeys class option if keyword
                              %display desired
\maketitle

%\tableofcontents

\section{\label{introduction}Introduction}

Introduction should be written down during the outline phase.


\begin{enumerate}
    \item Objectives
    \item Motivation
    \item Background
    \item Highlight: What should the reader pay attention to?
    \item Summary: What should the reader expect as a conclusion?
\end{enumerate}


\section{\label{background}Background}

\section{\label{results}Results and Discussions}


\begin{itemize}
\item Collect the key points and results.
\item Place figures and plots in the right order.
\item Show important results first.
\end{itemize}


George Whitesides addressed some writing styles.\cite{Whitesides2004}

\begin{enumerate}
    \item Put a noun right after the word \em{this}.
    \item Experimental results always past tense.
    \item When comparing, put the two objects being compared explicitly in one sentence. 
\end{enumerate}



\section{\label{conclusion}Conclusion}

\begin{itemize}
    \item Do not repeat what is in the results section.
    \item Higher level discussions.
    \item Show significance of this paper.
\end{itemize}



\medskip
 
\begin{thebibliography}{9}
\bibitem{Whitesides2004}
Whitesides, G. M.
\textit{Whitesides' Group: Writing a Paper}. 
Advanced Materials, 16(8):1375--1377, 2004.
\end{thebibliography}




\end{document}
%
% ****** End of file apssamp.tex ******
